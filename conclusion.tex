% Permission is granted to copy, distribute and/or modify this document
% under the terms of the GNU Free Documentation License, Version 1.3
% or any later version published by the Free Software Foundation;
% with no Invariant Sections, no Front-Cover Texts, and no Back-Cover Texts.
% A copy of the license is included in the section entitled "GNU
% Free Documentation License".
%
% Authors:
% Caner Candan <caner@candan.fr>, http://caner.candan.fr

\section{Conclusion}

Le concept de réutilisabilité a été intégré en partie grâce au librairie \petscx\ et \slepcx. Nous reconnaisons toutefois que la conception algorithmique des composants de base des méthodes numériques reste une étape laborieuse. Cela explique le peu de paramètres dans ces nouvelles librairies. Nous avons ainsi pu aborder toute au long de ce rapport les étapes d'installation et de configuration de l'environnement de travail, le nouveau design pour la réutilisabilité des composants de base, l'implémentation des opérations de base d'algèbre linéaire mais aussi et surtout les méthodes de calcul des valeurs propres en détaillant la méthode d'Arnoldi ainsi que les itératives ERAM et MERAM.\\

Ce papier suit les termes de la licence ``GNU Free Documentation License 1.3'' et peut être librement téléchargé\footnote{https://github.com/canercandan/reusability}, utilisé et modifié.
