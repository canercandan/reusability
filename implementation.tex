% Permission is granted to copy, distribute and/or modify this document
% under the terms of the GNU Free Documentation License, Version 1.3
% or any later version published by the Free Software Foundation;
% with no Invariant Sections, no Front-Cover Texts, and no Back-Cover Texts.
% A copy of the license is included in the section entitled "GNU
% Free Documentation License".
%
% Authors:
% Caner Candan <caner@candan.fr>, http://caner.candan.fr

\section{Implémentation}

Dans cette section nous allons voir comment implémenter des opérations matricielles de bases puis nous aborderons une implémentation de la méthode itérative ERAM et de la méthode hybride MERAM.

\paragraph{Opérations matricielles}

Le code \ref{ksp_matvec} illustre le produit matrice-vecteur sur le sous-espace de Krylov.

\begin{algorithm}[h]
  \caption{Exemple de code illustrant le produit matrice-vecteur sur le sous-espace de Krylov}
  \label{ksp_matvec}
\begin{verbatim}
#include <petsc_cxx/petsc_cxx>
using namespace petsc_cxx;
int main(int ac, char** av)
{
    Parser parser(ac, av);
    Context context(parser);
    const Int N = 4;
    Matrix< Scalar > A(N,N,10);
    Vector< Scalar > x(N,2);
    Vector< Scalar > b(N);
    Krylov< Scalar > ksp;
    ksp(A,x,b);
    return 0;
}
\end{verbatim}
\end{algorithm}

Nous utilisons la classe \verb#Krylov<T># implémenté dans la librairie \petscx. Cette classe appelle les fonctions de la structure \verb#KSP# disponible dans \petsc. Nous passons à l'instance \verb#ksp# de la classe, la matrice \verb#A# et les vecteurs \verb#x# et \verb#b#. Le resultat étant sauvegardé dans le vecteur \verb#b#.

\paragraph{EPS}

La librairie \slepc\ introduit une famille de fonctions, appelé \verb#EPS#\footnote{EPS: Eigen Problem Solver}, pour le calcul des valeurs propres. Il est indispensable de définir la méthode de résolution à utiliser parmi celles disponible:\\

\begin{itemize}
\item ``power'',
\item ``subspace'',
\item ``arnoldi'',
\item ``lanczos'',
\item ``krylovschur'',
\item ``lapack''
\item ``arpack''
\item ``blzpack'',
\item ``trlan'',
\item ``blopex'',
\item ``primme''\\
\end{itemize}

Nous utiliserons dans notre cas la méthode d'Arnoldi.


\paragraph{La projection d'Arnoldi}

Nous ne pourrons aborder la méthode d'Arnoldi sans décrire sa méthode de projection. L'algorithme \ref{projection} illustre ainsi la méthode de projection d'Arnoldi aussi appelé réduction d'Arnoldi.

\begin{algorithm}[h]
  \caption{Algorithme RA $(in:A,n,m,V,out:Hm,Vm)$}
  \label{projection}
\begin{enumerate}
\item Phase d'initialisation: Choix de $m$ et de $V (V_1 = \frac{V}{||V||})$
\item Phase de Projection:
\begin{verbatim}
Pour j = 1 à m faire
  z = AV_j
  Pour i = 1 à m faire
    h_ij = (z,v_i) //
    dot = Sigma(n,i=1)(x_i * y_i)
    z = z - h_ij * v_i
  Fin pour i

  h_j+1,j = || z ||
  v_j+1 = z / h_j+1,j
Fin pour j
\end{verbatim}

\item $Hm y_i = ~\lambda_i y_i, i|m,1$
\item $~u_i = V_m * y_i, i|m,1$

\end{enumerate}
\end{algorithm}

$Hm$ est une matrice d'Hessenberg inférieur.\footnote{\url{http://fr.wikipedia.org/wiki/Matrice_de_Hessenberg}}

\paragraph{Méthode d'Arnoldi}

L'algorithme \ref{ma} illustre les différentes étapes que constitue la méthode d'Arnoldi.

\begin{algorithm}[h]
  \caption{Algorithme MA $(in:A,n,s,m,V;out:\lambda_s,U_s)$}
  \label{ma}
  \begin{enumerate}
  \item Algorithme $RA(in:A,n,m,V;out:\lambda_m,U_m)$
  \item Résolution du probleme de valeur propre: $H_m.y_i = ~\lambda_i.y_i$ pour $i = 1$ à $m$ et selection de $s$ valeurs propres désirées
  \item Retour dans l'espace de départ: $~U_i = V_m.y_i$ pour $i = 1$ à $s$
  \end{enumerate}
\end{algorithm}

\slepc\ fournit un panel d'options pour les différentes méthodes de résolution notemment pour Arnoldi. Il est par exemple possible d'utiliser la méthode d'Arnoldi en mode différé avec l'option \verb#-eps_arnoldi_delayed#.

\paragraph{Méthode itérative ERAM}

Il\footnote{ERAM: Explicitely Restarted Arnoldi Method} s'agit d'un bon représentant des méthodes itératives d'algèbre linéaire. Elle contient toutes les operations élémentaires des méthodes itératives de Krylov.

L'algorithme \ref{eram} illustre les différentes étapes que constitue ERAM.

\begin{algorithm}[h]
  \caption{Algorithme ERAM $(in:A,n,s,l,M,V,Tol;out: 1_s, U_s, Rho_s)$}
  \label{eram}
  \begin{enumerate}
  \item Choix de $m,V,Tol$ \footnote{$Tol =$ valeur de tolérance}
  \item \label{eram_step2} Algorithme $MA(in: A,n,s,m,V;out:\lambda_s,U_s)$
  \item
\begin{verbatim}
Si rho_i = || A~u_i - ~lambda_i ~u_i ||\
   > Tol pour i = 1 à s alors
  V = sum^s_{i=1} alpha_i . ~u_i et\
  aller à l'étape 2.
Sinon Stop
\end{verbatim}
  \end{enumerate}
\end{algorithm}

La méthode de résolution d'Arnoldi, intégré à \slepc\, est une méthode ERAM. Le code \ref{eps} présente un exemple d'implémentation de la méthode ERAM en \slepcx.

\begin{algorithm}[h]
  \caption{Implémentation de ERAM en \slepcx}
  \label{eps}
\begin{verbatim}
#include <slepc_cxx/slepc_cxx>
typedef petsc_cxx::Scalar T;
int main(int ac, char** av)
{
    slepc_cxx::Parser parser(ac, av);
    petsc_cxx::Context context(parser);
    const Int N = 30;
    petsc_cxx::Matrix<T> A(N);

    // Quelques routines de remplissage
    // de la matrice creuse A
    // à l'aide des fonctions SLEPC.

    slepc_cxx::EPSolver<T> eps(EPSARNOLDI);
    eps(A);

    return 0;
}
\end{verbatim}
\end{algorithm}

\paragraph{Méthode hybride MERAM}

Une méthode hybride est une méthode définie par un ensemble de méthodes itératives qui collaborent afin d'accélérer la convergence d'une entre elles.\\

MERAM\footnote{MERAM: Multi-ERAM} est une méthode hybride définie par un ensemble d'instances d'une même méthode itératives (co-méthodes) qui collaborent afin d'accélerer la convergence d'une entre elles.\\

L'algorithme \ref{meram} illustre les différentes étapes que constitue MERAM.\\

\begin{algorithm}[h]
  \caption{Algorithme MERAM $(in:A,n,s,l,M,V,Tol;out: 1_s, U_s, Rho_s)$}
  \label{meram}
  \begin{enumerate}
  \item Projection
  \item Resolution dans le sous-espace
  \item Retour dans l'espace de départ
  \end{enumerate}
\end{algorithm}

On cherche $s$ valeurs et vecteurs propres d'une matrice creuse $A$ d'ordre $n$.

Le code \ref{eps_meram} présente un exemple d'implémentation de la méthode MERAM en \slepcx.

\begin{algorithm}[h]
  \caption{Implémentation de MERAM en \slepcx}
  \label{eps_meram}
\begin{verbatim}
#include <slepc_cxx/slepc_cxx>
typedef petsc_cxx::Scalar T;
int main(int ac, char** av)
{
    slepc_cxx::Parser parser(ac, av);
    petsc_cxx::Context context(parser);
    const Int N = 30;
    petsc_cxx::Matrix<T> A(N);

    // Quelques routines de remplissage
    // de la matrice creuse A
    // à l'aide des fonctions SLEPC.

    slepc_cxx::EPSolver<T> eps(EPSARNOLDI);
    slepc_cxx::MERAM<T> meram(eps);
    meram(A);

    return 0;
}
\end{verbatim}
\end{algorithm}

En reprenant le code d'implémentation de la méthode ERAM nous incluons l'appel à la classe MERAM dans la librairie \slepcx\ effectuant les différantes étapes cités précédement.
