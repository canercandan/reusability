% Permission is granted to copy, distribute and/or modify this document
% under the terms of the GNU Free Documentation License, Version 1.3
% or any later version published by the Free Software Foundation;
% with no Invariant Sections, no Front-Cover Texts, and no Back-Cover Texts.
% A copy of the license is included in the section entitled "GNU
% Free Documentation License".
%
% Authors:
% Caner Candan <caner@candan.fr>, http://caner.candan.fr

\section{Installation et configuration de l'environnement}

\paragraph{Préambule}

Il est important, avant tout, de présenter les démarches nécessaires à l'installation et la configuration de l'environnement. Nous entendons principalement par environnement les librairies ``PETSc'' et ``SLEPc''. Puis nous aborderons une méthode de configuration d'un projet utilisant ces librairies à l'aide d'outil comme ``CMake''.

\paragraph{Installation}

Utilisant un OS basé sur ``debian'', une ``Ubuntu 10.10'', il est possible d'installer ``PETSc'' et ``SLEPSc'', sans qu'aucune compilation soit nécessaire. En effet nous pouvons utiliser l'outil d'installation de paquets ``aptitute''. Une rapide introduction est présenté en figure \ref{fig:installation}.

\begin{figure}[here]
\begin{verbatim}
$> sudo aptitude install libpetsc3.1
$> sudo aptitude install libpetsc3.1-dev
$> sudo aptitude install petsc3.1-doc

$> sudo aptitude install libslepc3.1
$> sudo aptitude install libslepc3.1-dev
$> sudo aptitude install slepc3.1-doc
\end{verbatim}
  \caption{Commandes pour l'installation des librairies et des dépendances}
  \label{fig:installation}
\end{figure}

\paragraph{Configuration}

Puis pour éviter les ``Makefile'' désordonés et incomplètes fournit dans les exemples des librairies ainsi que les variables d'environnement à définir, nous pouvons faire appel à l'outil ``CMake''\footnote{CMake: générateur et language haut niveau de Makefile multi-plateforme}.\\

Ce dernier facilite grandement la configuration d'un projet en ``PETSc''\footnote{FindPETSc.cmake available from \url{https://github.com/jedbrown/cmake-modules/blob/master/FindPETSc.cmake}} ou ``SLEPc'' et permet ainsi à quiconque une rapide prise en main des bibliothèques utilisées. De plus la configuration de ``CMake'' est multi-plateforme. Le code \ref{cmake_project_hello} illustre un exemple de fichier de configuration ``CMake'' pour un projet basic.

\begin{algorithm}[h]
  \caption{Fichier de configuration d'un projet projet pour le programme ``hello world'' en PETSc avec CMake}
  \label{cmake_project_hello}
\begin{verbatim}
CMAKE_MINIMUM_REQUIRED(VERSION 2.8)
PROJECT(hello)
INCLUDE(FindPETSc)
INCLUDE_DIRECTORIES( ${PETSC_INCLUDES} )
ADD_EXECUTABLE(hello hello.cpp)
TARGET_LINK_LIBRARIES(hello ${PETSC_LIBRARIES})
\end{verbatim}
\end{algorithm}
