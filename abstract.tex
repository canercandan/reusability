% Permission is granted to copy, distribute and/or modify this document
% under the terms of the GNU Free Documentation License, Version 1.3
% or any later version published by the Free Software Foundation;
% with no Invariant Sections, no Front-Cover Texts, and no Back-Cover Texts.
% A copy of the license is included in the section entitled "GNU
% Free Documentation License".
%
% Authors:
% Caner Candan <caner@candan.fr>, http://caner.candan.fr

\begin{abstract}

Le Génie Logiciel constitue l'ensemble des activités permettant de définir la conception jusqu'à la production d'un logiciel et la Réutilisabilité peut être définit comme une manière de concevoir et réaliser les composants de base des méthodes numériques de façon la plus réutilisable possible. Pour les illustrer, nous utiliserons une méthode itérative, qui contrairement aux méthodes directes, se prête mieux aux matrices de grandes tailles. Elles permettent de réduire la taille de la matrice en se limitant aux éléments non-nuls. Il est également important de noter que la méthode itérative utilisée est une méthode de projection pour résoudre de très grandes tailles de problèmes. Elle est consitutée de 3 parties principales: la projection dans un sous-espace, la résolution avec une méthode classique et le retour dans l'espace de départ.\\

Nous rappellons, ci-dessous, les objectifs attendus du rapport:\\

\begin{itemize}
\item se familiariser avec la réutilisabilité séquentielle/parallèle à l’aide d’une bibliothèque numérique orientée composant,
\item utilisation des bibliothèques \petsc, \slepc,
\item implémentation des opérations matricielles de base, de la méthode itérative \eram\ et de la méthode hybride \meram,
\item mise à l'échelle des exemples réalisés en intégrant les composants de ces bibliothèques dans l'environnement \yml.
\end{itemize}

\end{abstract}
